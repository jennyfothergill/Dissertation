\chapter{Notebook Example for P3HT Validation}
\label{app:p3ht_nb}

%---------------------------------------------
\newpage % NOTE: The appendix title should be on its own page.
%---------------------------------------------

    \hypertarget{p3ht-validation}{%
\section{P3HT Validation}\label{p3ht-validation}}

    \begin{tcolorbox}[breakable, size=fbox, boxrule=1pt, pad at break*=1mm,colback=cellbackground, colframe=cellborder]
\prompt{In}{incolor}{1}{\boxspacing}
\begin{Verbatim}[commandchars=\\\{\}]
\PY{k+kn}{import} \PY{n+nn}{os}

\PY{k+kn}{from} \PY{n+nn}{cmeutils}\PY{n+nn}{.}\PY{n+nn}{structure} \PY{k+kn}{import} \PY{n}{order\PYZus{}parameter}
\PY{k+kn}{from} \PY{n+nn}{fresnel} \PY{k+kn}{import} \PY{n}{camera}\PY{p}{,} \PY{n}{pathtrace}\PY{p}{,} \PY{n}{light}
\PY{k+kn}{from} \PY{n+nn}{gixstapose}\PY{n+nn}{.}\PY{n+nn}{diffractometer} \PY{k+kn}{import} \PY{n}{Diffractometer}\PY{p}{,} \PY{n}{get\PYZus{}angle}
\PY{k+kn}{from} \PY{n+nn}{gixstapose}\PY{n+nn}{.}\PY{n+nn}{draw\PYZus{}scene} \PY{k+kn}{import} \PY{n}{get\PYZus{}scene}
\PY{k+kn}{import} \PY{n+nn}{gsd}\PY{n+nn}{.}\PY{n+nn}{hoomd}
\PY{k+kn}{import} \PY{n+nn}{matplotlib}\PY{n+nn}{.}\PY{n+nn}{pyplot} \PY{k}{as} \PY{n+nn}{plt}
\PY{k+kn}{import} \PY{n+nn}{numpy} \PY{k}{as} \PY{n+nn}{np}
\PY{k+kn}{from} \PY{n+nn}{PIL} \PY{k+kn}{import} \PY{n}{Image}
\PY{k+kn}{from} \PY{n+nn}{planckton}\PY{n+nn}{.}\PY{n+nn}{utils}\PY{n+nn}{.}\PY{n+nn}{units} \PY{k+kn}{import} \PY{n}{string\PYZus{}to\PYZus{}quantity}\PY{p}{,} \PY{n}{kelvin\PYZus{}from\PYZus{}reduced}
\PY{k+kn}{from} \PY{n+nn}{scipy}\PY{n+nn}{.}\PY{n+nn}{interpolate} \PY{k+kn}{import} \PY{n}{RectBivariateSpline}
\PY{k+kn}{import} \PY{n+nn}{signac}
\PY{k+kn}{import} \PY{n+nn}{unyt} \PY{k}{as} \PY{n+nn}{u}
\end{Verbatim}
\end{tcolorbox}

    In the following notebook, we will demonstrate the recreation of part a
of the following figure from
\href{http://www.mdpi.com/2073-4360/10/12/1305}{Miller 2018}. This
figure shows the order parameter of a OPLS-UA P3HT system at different
temperatures and solvent parameters (\(\epsilon_{s}\) or e\_factor).

    \begin{center}
    \adjustimage{max size={0.9\linewidth}{0.9\paperheight}}{figures/opa_notebook/Miller2018_fig3.png}
    \end{center}

The order parameter is used to quantify the degree of ordering in the
system by grouping structures into clusters, and then calculating the
ratio of the structures in a ``large'' clusters over the total number of
structures in the system. In the case of a P3HT polymer, the structure
we use for the order parameter analysis is the thiophene moeity. The
clustering criteria depends on the angle between the planes of the
thiophenes and the distance between the thiophene centers. A thiophene
pair must meet both clustering criteria to be considered cluster
neighbors. For example, if the angle cutoff is 10° and the distance
cutoff is 6\AA, the image below shows a thiophene pair which does not meet
the criteria (encircled and crossed in red) and another which does
(encircled in green).

    \begin{center}
    \adjustimage{max size={0.9\linewidth}{0.9\paperheight}}{figures/opa_notebook/p3ht-alignment.png}
    \end{center}

For a cluster to be considered ``large'', it must contain at least 6
thiophenes.

All of this analysis, from the simulations to the functions used to run
the analysis itself has been redone using new tools. The goal is to show
that not only can we recreate our previous work, but also the tools
we've developed make this analysis more transparent, reproducible,
usable, and extensible.

The dataset used in this notebook was generated using
\href{https://github.com/cmelab/planckton-flow}{PlanckTon}, which
provides a way to easily and reproducibly interface with the
\href{https://hoomd-blue.readthedocs.io/en/latest/index.html}{HOOMD-blue}
molecular dynamics engine, and
\href{https://github.com/cmelab/planckton-flow}{PlanckTon-flow}, which
uses \href{https://docs.signac.io/en/latest/}{Signac} to initialize and
submit simulations across the desired parameter space. The workspace
with all data used in this notebook can be found
\href{https://zenodo.org/record/5911940}{here}.

First, assuming the tarball from the above link is unpacked in this
directory, we can use signac to explore this workspace:

    \begin{tcolorbox}[breakable, size=fbox, boxrule=1pt, pad at break*=1mm,colback=cellbackground, colframe=cellborder]
\prompt{In}{incolor}{2}{\boxspacing}
\begin{Verbatim}[commandchars=\\\{\}]
\PY{n}{p} \PY{o}{=} \PY{n}{signac}\PY{o}{.}\PY{n}{get\PYZus{}project}\PY{p}{(}\PY{l+s+s2}{\PYZdq{}}\PY{l+s+s2}{p3ht\PYZhy{}ua}\PY{l+s+s2}{\PYZdq{}}\PY{p}{)}

\PY{n}{p}\PY{o}{.}\PY{n}{detect\PYZus{}schema}\PY{p}{(}\PY{p}{)}
\end{Verbatim}
\end{tcolorbox}

            \begin{tcolorbox}[breakable, size=fbox, boxrule=.5pt, pad at break*=1mm, opacityfill=0]
\prompt{Out}{outcolor}{2}{\boxspacing}
\begin{Verbatim}[commandchars=\\\{\}]
ProjectSchema(<len=15>)
\{
 'density': 'str([0.56_g-cm**3], 1)',
 'dt': 'float([0.001], 1)',
 'e_factor': 'float([0.2, 0.4, 0.6, 0.8, 1.0], 5)',
 'forcefield': 'str([gaff-custom], 1)',
 'input': 'tuple([('P3HT-16-gaff',)], 1)',
 'kT': 'tuple([(1,), (1.5,), (2,), ..., (4,), (5,)], 8)',
 'mode': 'str([gpu], 1)',
 'n_compounds': 'tuple([(100,)], 1)',
 'n_steps': 'tuple([(100000000.0,)], 1)',
 'r_cut': 'float([2.5], 1)',
 'remove_hydrogens': 'bool([True], 1)',
 'shrink_kT': 'int([5], 1)',
 'shrink_steps': 'float([100000.0], 1)',
 'shrink_tau': 'int([1], 1)',
 'tau': 'tuple([(0.3,)], 1)',
\}
\end{Verbatim}
\end{tcolorbox}
        
    The above schema shows that we are looking at a dataspace of P3HT
16-mers, at a density of 0.56 \(g/cm^{3}\) at 8 different temperatures
and 5 different e\_factors. In order to run the order parameter
analysis, there is a little bookkeeping to be done with units.

PlanckTon runs its simulations in
\href{https://hoomd-blue.readthedocs.io/en/latest/units.html}{reduced
units} to reduce floating point error. This means all lengths are scaled
such that the largest sigma value in the non-bonded potential is equal
to 1, and so on with the largest epsilon value in the non-bonded
potential, and the largest particle mass.

This information is stored in the
\href{https://docs.signac.io/en/latest/jobs.html\#the-job-document}{job
document} and can be converted to a
\href{https://unyt.readthedocs.io/en/stable/}{unyt quantity} using the
\texttt{string\_to\_quantity} function.

In the cell below we set our angle and distance cutoffs for the order
parameter analysis to 10° and 6\AA~ scaled by the unit length of our
simulation.

    \begin{tcolorbox}[breakable, size=fbox, boxrule=1pt, pad at break*=1mm,colback=cellbackground, colframe=cellborder]
\prompt{In}{incolor}{3}{\boxspacing}
\begin{Verbatim}[commandchars=\\\{\}]
\PY{k}{for} \PY{n}{job} \PY{o+ow}{in} \PY{n}{p}\PY{p}{:}
    \PY{k}{if} \PY{n}{job}\PY{o}{.}\PY{n}{doc}\PY{o}{.}\PY{n}{get}\PY{p}{(}\PY{l+s+s2}{\PYZdq{}}\PY{l+s+s2}{done}\PY{l+s+s2}{\PYZdq{}}\PY{p}{)}\PY{p}{:}
        \PY{k}{break}
        
\PY{n}{ref\PYZus{}energy} \PY{o}{=} \PY{n}{string\PYZus{}to\PYZus{}quantity}\PY{p}{(}\PY{n}{job}\PY{o}{.}\PY{n}{doc}\PY{p}{[}\PY{l+s+s2}{\PYZdq{}}\PY{l+s+s2}{ref\PYZus{}energy}\PY{l+s+s2}{\PYZdq{}}\PY{p}{]}\PY{p}{)}
\PY{n}{ref\PYZus{}mass} \PY{o}{=} \PY{n}{string\PYZus{}to\PYZus{}quantity}\PY{p}{(}\PY{n}{job}\PY{o}{.}\PY{n}{doc}\PY{p}{[}\PY{l+s+s2}{\PYZdq{}}\PY{l+s+s2}{ref\PYZus{}mass}\PY{l+s+s2}{\PYZdq{}}\PY{p}{]}\PY{p}{)}
\PY{n}{ref\PYZus{}distance} \PY{o}{=} \PY{n}{string\PYZus{}to\PYZus{}quantity}\PY{p}{(}\PY{n}{job}\PY{o}{.}\PY{n}{doc}\PY{p}{[}\PY{l+s+s2}{\PYZdq{}}\PY{l+s+s2}{ref\PYZus{}distance}\PY{l+s+s2}{\PYZdq{}}\PY{p}{]}\PY{p}{)}
\PY{n+nb}{print}\PY{p}{(}\PY{l+s+sa}{f}\PY{l+s+s2}{\PYZdq{}}\PY{l+s+si}{\PYZob{}}\PY{n}{ref\PYZus{}distance}\PY{l+s+si}{:}\PY{l+s+s2}{.3f}\PY{l+s+si}{\PYZcb{}}\PY{l+s+s2}{ }\PY{l+s+si}{\PYZob{}}\PY{n}{ref\PYZus{}mass}\PY{l+s+si}{:}\PY{l+s+s2}{.2e}\PY{l+s+si}{\PYZcb{}}\PY{l+s+s2}{ }\PY{l+s+si}{\PYZob{}}\PY{n}{ref\PYZus{}energy}\PY{l+s+si}{:}\PY{l+s+s2}{.3f}\PY{l+s+si}{\PYZcb{}}\PY{l+s+s2}{\PYZdq{}}\PY{p}{)}

\PY{n}{r\PYZus{}max} \PY{o}{=} \PY{n+nb}{float}\PY{p}{(}\PY{l+m+mi}{6} \PY{o}{*} \PY{n}{u}\PY{o}{.}\PY{n}{Angstrom} \PY{o}{/} \PY{n}{ref\PYZus{}distance}\PY{p}{)}
\PY{n}{a\PYZus{}max} \PY{o}{=} \PY{l+m+mi}{10}
\PY{n+nb}{print}\PY{p}{(}\PY{l+s+sa}{f}\PY{l+s+s2}{\PYZdq{}}\PY{l+s+s2}{r\PYZus{}max: }\PY{l+s+si}{\PYZob{}}\PY{n}{r\PYZus{}max}\PY{l+s+si}{:}\PY{l+s+s2}{.2f}\PY{l+s+si}{\PYZcb{}}\PY{l+s+s2}{, a\PYZus{}max: }\PY{l+s+si}{\PYZob{}}\PY{n}{a\PYZus{}max}\PY{l+s+si}{\PYZcb{}}\PY{l+s+s2}{\PYZdq{}}\PY{p}{)}
\end{Verbatim}
\end{tcolorbox}

    \begin{Verbatim}[commandchars=\\\{\}]
3.564 \AA~ 5.32e-26 kg 0.250 kcal/mol
r\_max: 1.68, a\_max: 10
    \end{Verbatim}

    In the following cell, the order parameter analysis is performed over
the entire workspace. Previously, this was done using
\href{https://bitbucket.org/cmelab/evan_analysis_scripts/src/master/order_over_time.py}{manual
indexing and was hardcoded for our P3HT input files}. But this workflow
has been updated to use \href{https://github.com/cmelab/grits}{GRiTS}
which uses
\href{https://www.daylight.com/dayhtml/doc/theory/theory.smarts.html}{SMARTS}
chemical grammar to detect chemical patterns and map the particles in
that pattern to a bead. The \texttt{order\_parameter} function in
\href{https://github.com/cmelab/cmeutils}{cmeutils} then takes this
mapping and the bead positions from GRiTS and calculates the order
parameter of the last 10 frames of the trajectory.

    \begin{tcolorbox}[breakable, size=fbox, boxrule=1pt, pad at break*=1mm,colback=cellbackground, colframe=cellborder]
\prompt{In}{incolor}{4}{\boxspacing}
\begin{Verbatim}[commandchars=\\\{\}]
\PY{o}{\PYZpc{}\PYZpc{}time}

\PY{n}{o\PYZus{}dict} \PY{o}{=} \PY{p}{\PYZob{}}\PY{p}{\PYZcb{}}
\PY{n}{orders} \PY{o}{=} \PY{p}{[}\PY{p}{]}
\PY{n}{job\PYZus{}strs} \PY{o}{=} \PY{p}{[}\PY{p}{]}
\PY{n}{ts} \PY{o}{=} \PY{p}{[}\PY{p}{]}
\PY{n}{efs} \PY{o}{=} \PY{p}{[}\PY{p}{]}
\PY{n}{keys} \PY{o}{=} \PY{p}{[}\PY{p}{]}
\PY{k}{for} \PY{p}{(}\PY{n}{ef}\PY{p}{,} \PY{n}{kt}\PY{p}{)}\PY{p}{,} \PY{n}{jobs} \PY{o+ow}{in} \PY{n}{p}\PY{o}{.}\PY{n}{groupby}\PY{p}{(}\PY{p}{[}\PY{l+s+s2}{\PYZdq{}}\PY{l+s+s2}{e\PYZus{}factor}\PY{l+s+s2}{\PYZdq{}}\PY{p}{,} \PY{l+s+s2}{\PYZdq{}}\PY{l+s+s2}{kT}\PY{l+s+s2}{\PYZdq{}}\PY{p}{]}\PY{p}{)}\PY{p}{:}
    \PY{k}{for} \PY{n}{job} \PY{o+ow}{in} \PY{n}{jobs}\PY{p}{:}
        \PY{k}{if} \PY{n}{job}\PY{o}{.}\PY{n}{doc}\PY{o}{.}\PY{n}{get}\PY{p}{(}\PY{l+s+s2}{\PYZdq{}}\PY{l+s+s2}{done}\PY{l+s+s2}{\PYZdq{}}\PY{p}{)}\PY{p}{:}
            \PY{n}{t\PYZus{}si} \PY{o}{=} \PY{n+nb}{int}\PY{p}{(}\PY{n}{kelvin\PYZus{}from\PYZus{}reduced}\PY{p}{(}\PY{n}{kt}\PY{p}{[}\PY{l+m+mi}{0}\PY{p}{]}\PY{p}{,}\PY{n}{ref\PYZus{}energy}\PY{p}{)}\PY{p}{)}
            \PY{n}{key} \PY{o}{=} \PY{p}{(}\PY{n}{ef}\PY{p}{,} \PY{n}{t\PYZus{}si}\PY{p}{)}
            \PY{k}{if} \PY{n}{key} \PY{o+ow}{not} \PY{o+ow}{in} \PY{n}{keys}\PY{p}{:}
                \PY{n}{gsdfile} \PY{o}{=} \PY{n}{job}\PY{o}{.}\PY{n}{fn}\PY{p}{(}\PY{l+s+s2}{\PYZdq{}}\PY{l+s+s2}{trajectory.gsd}\PY{l+s+s2}{\PYZdq{}}\PY{p}{)}
                \PY{n}{cg\PYZus{}gsdfile} \PY{o}{=} \PY{n}{job}\PY{o}{.}\PY{n}{fn}\PY{p}{(}\PY{l+s+s2}{\PYZdq{}}\PY{l+s+s2}{cg\PYZhy{}trajectory.gsd}\PY{l+s+s2}{\PYZdq{}}\PY{p}{)}
                \PY{n}{mapfile} \PY{o}{=} \PY{n}{job}\PY{o}{.}\PY{n}{fn}\PY{p}{(}\PY{l+s+s2}{\PYZdq{}}\PY{l+s+s2}{mapping.json}\PY{l+s+s2}{\PYZdq{}}\PY{p}{)}
                \PY{k}{if} \PY{n}{job}\PY{o}{.}\PY{n}{doc}\PY{o}{.}\PY{n}{get}\PY{p}{(}\PY{l+s+s2}{\PYZdq{}}\PY{l+s+s2}{order}\PY{l+s+s2}{\PYZdq{}}\PY{p}{)} \PY{o+ow}{is} \PY{k+kc}{None}\PY{p}{:}
                    \PY{k}{if} \PY{o+ow}{not} \PY{p}{(}\PY{n}{exists}\PY{p}{(}\PY{n}{mapfile}\PY{p}{)} \PY{o+ow}{and} \PY{n}{exists}\PY{p}{(}\PY{n}{cg\PYZus{}gsdfile}\PY{p}{)}\PY{p}{)}\PY{p}{:}
                        \PY{n}{system} \PY{o}{=} \PY{n}{CG\PYZus{}System}\PY{p}{(}
                            \PY{n}{gsdfile}\PY{p}{,} 
                            \PY{n}{beads}\PY{o}{=}\PY{p}{\PYZob{}}\PY{l+s+s2}{\PYZdq{}}\PY{l+s+s2}{\PYZus{}B}\PY{l+s+s2}{\PYZdq{}} \PY{p}{:} \PY{l+s+s2}{\PYZdq{}}\PY{l+s+s2}{c1cscc1}\PY{l+s+s2}{\PYZdq{}}\PY{p}{\PYZcb{}}\PY{p}{,} 
                            \PY{n}{conversion\PYZus{}dict}\PY{o}{=}\PY{n}{amber\PYZus{}dict}\PY{p}{,}
                            \PY{n}{add\PYZus{}hydrogens}\PY{o}{=}\PY{k+kc}{True}\PY{p}{,}
                        \PY{p}{)}
                        \PY{n}{mapping} \PY{o}{=} \PY{n}{system}\PY{o}{.}\PY{n}{mapping}\PY{p}{[}\PY{l+s+s2}{\PYZdq{}}\PY{l+s+s2}{\PYZus{}B...c1cscc1}\PY{l+s+s2}{\PYZdq{}}\PY{p}{]}
                        \PY{n}{system}\PY{o}{.}\PY{n}{save\PYZus{}mapping}\PY{p}{(}\PY{n}{mapfile}\PY{p}{)}
                        \PY{n}{system}\PY{o}{.}\PY{n}{save}\PY{p}{(}\PY{n}{cg\PYZus{}gsdfile}\PY{p}{)}
                        \PY{n+nb}{print}\PY{p}{(}\PY{l+s+s2}{\PYZdq{}}\PY{l+s+se}{\PYZbs{}t}\PY{l+s+s2}{CG\PYZus{}System created}\PY{l+s+s2}{\PYZdq{}}\PY{p}{)}
                    \PY{k}{else}\PY{p}{:}
                        \PY{k}{with} \PY{n+nb}{open}\PY{p}{(}\PY{n}{mapfile}\PY{p}{)} \PY{k}{as} \PY{n}{f}\PY{p}{:}
                            \PY{n}{d} \PY{o}{=} \PY{n}{json}\PY{o}{.}\PY{n}{load}\PY{p}{(}\PY{n}{f}\PY{p}{)}
                        \PY{n}{mapping} \PY{o}{=} \PY{n}{np}\PY{o}{.}\PY{n}{stack}\PY{p}{(}\PY{n}{d}\PY{p}{[}\PY{l+s+s2}{\PYZdq{}}\PY{l+s+s2}{\PYZus{}B...c1cscc1}\PY{l+s+s2}{\PYZdq{}}\PY{p}{]}\PY{p}{)}
                        \PY{n+nb}{print}\PY{p}{(}\PY{l+s+s2}{\PYZdq{}}\PY{l+s+se}{\PYZbs{}t}\PY{l+s+s2}{Using mapping}\PY{l+s+s2}{\PYZdq{}}\PY{p}{)}
                    \PY{n}{order}\PY{p}{,} \PY{n}{\PYZus{}} \PY{o}{=} \PY{n}{order\PYZus{}parameter}\PY{p}{(}\PY{n}{gsdfile}\PY{p}{,} \PY{n}{cg\PYZus{}gsdfile}\PY{p}{,} \PY{n}{mapping}\PY{p}{,} \PY{n}{r\PYZus{}max}\PY{p}{,} \PY{n}{a\PYZus{}max}\PY{p}{)}
                    \PY{n}{order} \PY{o}{=} \PY{n}{np}\PY{o}{.}\PY{n}{mean}\PY{p}{(}\PY{n}{order}\PY{p}{)}
                    \PY{n}{job}\PY{o}{.}\PY{n}{doc}\PY{p}{[}\PY{l+s+s2}{\PYZdq{}}\PY{l+s+s2}{order}\PY{l+s+s2}{\PYZdq{}}\PY{p}{]} \PY{o}{=} \PY{n}{order}
                \PY{k}{else}\PY{p}{:}
                    \PY{n}{order} \PY{o}{=} \PY{n}{job}\PY{o}{.}\PY{n}{doc}\PY{o}{.}\PY{n}{order}
                \PY{n}{o\PYZus{}dict}\PY{p}{[}\PY{n}{key}\PY{p}{]} \PY{o}{=} \PY{n}{order}
                \PY{n}{orders}\PY{o}{.}\PY{n}{append}\PY{p}{(}\PY{n}{order}\PY{p}{)}
                \PY{n}{keys}\PY{o}{.}\PY{n}{append}\PY{p}{(}\PY{n}{key}\PY{p}{)}
                \PY{n}{ts}\PY{o}{.}\PY{n}{append}\PY{p}{(}\PY{n}{t\PYZus{}si}\PY{p}{)}
                \PY{n}{efs}\PY{o}{.}\PY{n}{append}\PY{p}{(}\PY{n}{ef}\PY{p}{)}
                \PY{n}{job\PYZus{}strs}\PY{o}{.}\PY{n}{append}\PY{p}{(}\PY{n+nb}{str}\PY{p}{(}\PY{n}{job}\PY{p}{)}\PY{p}{)}
\end{Verbatim}
\end{tcolorbox}

    \begin{Verbatim}[commandchars=\\\{\}]
CPU times: user 29.2 ms, sys: 2.94 ms, total: 32.2 ms
Wall time: 30 ms
    \end{Verbatim}

    Next we will plot the order parameter vs T and e\_factor using an
\href{https://docs.scipy.org/doc/scipy/reference/generated/scipy.interpolate.RectBivariateSpline.html\#scipy-interpolate-rectbivariatespline}{interpolating
function from scipy}.

    \begin{tcolorbox}[breakable, size=fbox, boxrule=1pt, pad at break*=1mm,colback=cellbackground, colframe=cellborder]
\prompt{In}{incolor}{5}{\boxspacing}
\begin{Verbatim}[commandchars=\\\{\}]
\PY{n}{x} \PY{o}{=} \PY{n}{np}\PY{o}{.}\PY{n}{array}\PY{p}{(}\PY{n}{efs}\PY{p}{)}
\PY{n}{y} \PY{o}{=} \PY{n}{np}\PY{o}{.}\PY{n}{array}\PY{p}{(}\PY{n}{ts}\PY{p}{)}
\PY{n}{z} \PY{o}{=} \PY{n}{np}\PY{o}{.}\PY{n}{array}\PY{p}{(}\PY{n}{orders}\PY{p}{)}

\PY{n}{ux} \PY{o}{=} \PY{n}{np}\PY{o}{.}\PY{n}{unique}\PY{p}{(}\PY{n}{x}\PY{p}{)}
\PY{n}{uy} \PY{o}{=} \PY{n}{np}\PY{o}{.}\PY{n}{unique}\PY{p}{(}\PY{n}{y}\PY{p}{)}
\PY{n}{uz} \PY{o}{=} \PY{n}{np}\PY{o}{.}\PY{n}{zeros}\PY{p}{(}\PY{p}{(}\PY{n+nb}{len}\PY{p}{(}\PY{n}{uy}\PY{p}{)}\PY{p}{,}\PY{n+nb}{len}\PY{p}{(}\PY{n}{ux}\PY{p}{)}\PY{p}{)}\PY{p}{)}

\PY{c+c1}{\PYZsh{} key = (ef, t\PYZus{}si)}
\PY{k}{for} \PY{n}{i}\PY{p}{,} \PY{n}{xi} \PY{o+ow}{in} \PY{n+nb}{enumerate}\PY{p}{(}\PY{n}{ux}\PY{p}{)}\PY{p}{:}
    \PY{k}{for} \PY{n}{j}\PY{p}{,} \PY{n}{yj} \PY{o+ow}{in} \PY{n+nb}{enumerate}\PY{p}{(}\PY{n}{uy}\PY{p}{)}\PY{p}{:}
        \PY{n}{uz}\PY{p}{[}\PY{n}{j}\PY{p}{,}\PY{n}{i}\PY{p}{]} \PY{o}{=} \PY{n}{o\PYZus{}dict}\PY{p}{[}\PY{p}{(}\PY{n}{xi}\PY{p}{,} \PY{n}{yj}\PY{p}{)}\PY{p}{]}

\PY{n}{f} \PY{o}{=} \PY{n}{RectBivariateSpline}\PY{p}{(}\PY{n}{uy}\PY{p}{,}\PY{n}{ux}\PY{p}{,}\PY{n}{uz}\PY{p}{)}

\PY{n}{xs} \PY{o}{=} \PY{n}{np}\PY{o}{.}\PY{n}{linspace}\PY{p}{(}\PY{l+m+mf}{0.2}\PY{p}{,} \PY{l+m+mf}{1.2}\PY{p}{,} \PY{l+m+mi}{15}\PY{p}{)}
\PY{n}{ys} \PY{o}{=} \PY{n}{np}\PY{o}{.}\PY{n}{linspace}\PY{p}{(}\PY{l+m+mi}{50}\PY{p}{,} \PY{l+m+mi}{700}\PY{p}{,} \PY{l+m+mi}{15}\PY{p}{)}
\PY{n}{zs} \PY{o}{=} \PY{n}{f}\PY{p}{(}\PY{n}{ys}\PY{p}{,} \PY{n}{xs}\PY{p}{)}

\PY{n}{zs}\PY{p}{[}\PY{n}{np}\PY{o}{.}\PY{n}{where}\PY{p}{(}\PY{n}{zs} \PY{o}{\PYZgt{}} \PY{l+m+mi}{1}\PY{p}{)}\PY{p}{]} \PY{o}{=} \PY{l+m+mi}{1}
\PY{n}{zs}\PY{p}{[}\PY{n}{np}\PY{o}{.}\PY{n}{where}\PY{p}{(}\PY{n}{zs} \PY{o}{\PYZlt{}} \PY{l+m+mi}{0}\PY{p}{)}\PY{p}{]} \PY{o}{=} \PY{l+m+mi}{0}

\PY{n}{plt}\PY{o}{.}\PY{n}{contourf}\PY{p}{(}\PY{n}{xs}\PY{p}{,} \PY{n}{ys}\PY{p}{,} \PY{n}{zs}\PY{p}{,} \PY{l+m+mi}{15}\PY{p}{,} \PY{n}{cmap}\PY{o}{=}\PY{l+s+s2}{\PYZdq{}}\PY{l+s+s2}{rainbow}\PY{l+s+s2}{\PYZdq{}}\PY{p}{,} \PY{n}{vmax}\PY{o}{=}\PY{l+m+mi}{1}\PY{p}{,} \PY{n}{vmin}\PY{o}{=}\PY{l+m+mi}{0}\PY{p}{)}

\PY{n}{cbar} \PY{o}{=} \PY{n}{plt}\PY{o}{.}\PY{n}{colorbar}\PY{p}{(}\PY{n}{ticks}\PY{o}{=}\PY{n}{np}\PY{o}{.}\PY{n}{linspace}\PY{p}{(}\PY{l+m+mi}{0}\PY{p}{,} \PY{l+m+mi}{1}\PY{p}{,} \PY{l+m+mi}{5}\PY{p}{,} \PY{n}{endpoint}\PY{o}{=}\PY{k+kc}{True}\PY{p}{)}\PY{p}{)}
\PY{n}{cbar}\PY{o}{.}\PY{n}{ax}\PY{o}{.}\PY{n}{set\PYZus{}title}\PY{p}{(}\PY{l+s+sa}{r}\PY{l+s+s2}{\PYZdq{}}\PY{l+s+s2}{\PYZdl{}}\PY{l+s+s2}{\PYZbs{}}\PY{l+s+s2}{psi\PYZdl{}}\PY{l+s+s2}{\PYZdq{}}\PY{p}{,} \PY{n}{fontsize}\PY{o}{=}\PY{l+m+mi}{40}\PY{p}{,} \PY{n}{pad}\PY{o}{=}\PY{l+m+mi}{20}\PY{p}{)}

\PY{c+c1}{\PYZsh{} Show the positions of the sample points, just to have some reference}
\PY{n}{plt}\PY{o}{.}\PY{n}{scatter}\PY{p}{(}\PY{n}{x}\PY{p}{,} \PY{n}{y}\PY{p}{,} \PY{n}{c}\PY{o}{=}\PY{l+s+s2}{\PYZdq{}}\PY{l+s+s2}{k}\PY{l+s+s2}{\PYZdq{}}\PY{p}{,} \PY{n}{marker}\PY{o}{=}\PY{l+s+s2}{\PYZdq{}}\PY{l+s+s2}{x}\PY{l+s+s2}{\PYZdq{}}\PY{p}{,} \PY{n}{s}\PY{o}{=}\PY{l+m+mi}{100}\PY{p}{)}
\PY{n}{plt}\PY{o}{.}\PY{n}{xlabel}\PY{p}{(}\PY{l+s+sa}{r}\PY{l+s+s2}{\PYZdq{}}\PY{l+s+s2}{\PYZdl{}}\PY{l+s+s2}{\PYZbs{}}\PY{l+s+s2}{epsilon\PYZus{}}\PY{l+s+si}{\PYZob{}s\PYZcb{}}\PY{l+s+s2}{\PYZdl{}}\PY{l+s+s2}{\PYZdq{}}\PY{p}{)}
\PY{n}{plt}\PY{o}{.}\PY{n}{ylabel}\PY{p}{(}\PY{l+s+s2}{\PYZdq{}}\PY{l+s+s2}{Temperature (K)}\PY{l+s+s2}{\PYZdq{}}\PY{p}{)}
\PY{n}{plt}\PY{o}{.}\PY{n}{xticks}\PY{p}{(}\PY{n}{np}\PY{o}{.}\PY{n}{linspace}\PY{p}{(}\PY{l+m+mf}{0.2}\PY{p}{,} \PY{l+m+mf}{1.2}\PY{p}{,} \PY{l+m+mi}{6}\PY{p}{)}\PY{p}{)}
\PY{n}{fig} \PY{o}{=} \PY{n}{plt}\PY{o}{.}\PY{n}{gcf}\PY{p}{(}\PY{p}{)}
\PY{n}{fig}\PY{o}{.}\PY{n}{set\PYZus{}size\PYZus{}inches}\PY{p}{(}\PY{l+m+mi}{12}\PY{p}{,} \PY{l+m+mi}{10}\PY{p}{)}
\PY{n}{plt}\PY{o}{.}\PY{n}{savefig}\PY{p}{(}\PY{l+s+s2}{\PYZdq{}}\PY{l+s+s2}{order\PYZus{}parameter.pdf}\PY{l+s+s2}{\PYZdq{}}\PY{p}{)}
\end{Verbatim}
\end{tcolorbox}

    \begin{center}
    \adjustimage{max size={0.9\linewidth}{0.9\paperheight}}{figures/opa_notebook/Order_parameter_analysis_9_1.png}
    \end{center}
    
    Even with the changes to our analysis method (using GRiTS instead of
manual indexing), the forcefield (using a GAFF-UA model with flexible
thiophenes instead of an OPLS-UA model with rigid thiophenes), and the
clustering criteria (10° instead of 20°), the trend in the order
parameter across temperature and e\_factor space appears pretty robust!
I have included only part a from Miller 2018 below with the plot skewed
so that the bounds are the same.

    \begin{center}
    \adjustimage{max size={0.9\linewidth}{0.9\paperheight}}{figures/opa_notebook/Miller2018_fig3_aonly.png}
    \end{center}

We can show that our order parameter metric is robust, but how does it
relate to any physically measurable analysis? To address this question,
let's look at some diffraction patterns for the highest and lowest order
jobs to see if our order parameter metric is correlated with differences
in the periodic structure. We'll be using
\href{https://github.com/cmelab/GIXStapose}{GIXStapose} to visualize the
real space structure and its simulated diffraction pattern.

    \begin{tcolorbox}[breakable, size=fbox, boxrule=1pt, pad at break*=1mm,colback=cellbackground, colframe=cellborder]
\prompt{In}{incolor}{6}{\boxspacing}
\begin{Verbatim}[commandchars=\\\{\}]
\PY{n}{highest\PYZus{}order\PYZus{}jobs} \PY{o}{=} \PY{p}{[}\PY{n}{job\PYZus{}strs}\PY{p}{[}\PY{n}{orders}\PY{o}{.}\PY{n}{index}\PY{p}{(}\PY{n}{i}\PY{p}{)}\PY{p}{]} \PY{k}{for} \PY{n}{i} \PY{o+ow}{in} \PY{n+nb}{sorted}\PY{p}{(}\PY{n}{orders}\PY{p}{)}\PY{p}{[}\PY{o}{\PYZhy{}}\PY{l+m+mi}{3}\PY{p}{:}\PY{p}{]}\PY{p}{]}
\PY{n}{lowest\PYZus{}order\PYZus{}jobs} \PY{o}{=} \PY{p}{[}\PY{n}{job\PYZus{}strs}\PY{p}{[}\PY{n}{orders}\PY{o}{.}\PY{n}{index}\PY{p}{(}\PY{n}{i}\PY{p}{)}\PY{p}{]} \PY{k}{for} \PY{n}{i} \PY{o+ow}{in} \PY{n+nb}{sorted}\PY{p}{(}\PY{n}{orders}\PY{p}{)}\PY{p}{[}\PY{p}{:}\PY{l+m+mi}{4}\PY{p}{]}\PY{p}{]}

\PY{n+nb}{print}\PY{p}{(}\PY{l+s+s2}{\PYZdq{}}\PY{l+s+s2}{Higest order}\PY{l+s+s2}{\PYZdq{}}\PY{p}{)}
\PY{k}{for} \PY{n}{job\PYZus{}id} \PY{o+ow}{in} \PY{n}{highest\PYZus{}order\PYZus{}jobs}\PY{p}{:}
    \PY{n}{job} \PY{o}{=} \PY{n}{p}\PY{o}{.}\PY{n}{open\PYZus{}job}\PY{p}{(}\PY{n+nb}{id}\PY{o}{=}\PY{n}{job\PYZus{}id}\PY{p}{)}
    \PY{n}{T\PYZus{}SI} \PY{o}{=} \PY{n+nb}{int}\PY{p}{(}\PY{n}{kelvin\PYZus{}from\PYZus{}reduced}\PY{p}{(}\PY{n}{job}\PY{o}{.}\PY{n}{sp}\PY{o}{.}\PY{n}{kT}\PY{p}{[}\PY{l+m+mi}{0}\PY{p}{]}\PY{p}{,}\PY{n}{ref\PYZus{}energy}\PY{p}{)}\PY{p}{)}
    \PY{n+nb}{print}\PY{p}{(}\PY{l+s+sa}{f}\PY{l+s+s2}{\PYZdq{}}\PY{l+s+s2}{e\PYZus{}factor: }\PY{l+s+si}{\PYZob{}}\PY{n}{job}\PY{o}{.}\PY{n}{sp}\PY{o}{.}\PY{n}{e\PYZus{}factor}\PY{l+s+si}{\PYZcb{}}\PY{l+s+s2}{, T: }\PY{l+s+si}{\PYZob{}}\PY{n}{T\PYZus{}SI}\PY{l+s+si}{\PYZcb{}}\PY{l+s+s2}{K}\PY{l+s+s2}{\PYZdq{}}\PY{p}{)}
    \PY{n+nb}{print}\PY{p}{(}\PY{l+s+sa}{f}\PY{l+s+s2}{\PYZdq{}}\PY{l+s+se}{\PYZbs{}t}\PY{l+s+si}{\PYZob{}}\PY{n}{job}\PY{l+s+si}{\PYZcb{}}\PY{l+s+s2}{\PYZdq{}}\PY{p}{)}
    
\PY{n+nb}{print}\PY{p}{(}\PY{l+s+s2}{\PYZdq{}}\PY{l+s+se}{\PYZbs{}n}\PY{l+s+se}{\PYZbs{}n}\PY{l+s+s2}{Lowest order}\PY{l+s+s2}{\PYZdq{}}\PY{p}{)}
\PY{k}{for} \PY{n}{job\PYZus{}id} \PY{o+ow}{in} \PY{n}{lowest\PYZus{}order\PYZus{}jobs}\PY{p}{:}
    \PY{n}{job} \PY{o}{=} \PY{n}{p}\PY{o}{.}\PY{n}{open\PYZus{}job}\PY{p}{(}\PY{n+nb}{id}\PY{o}{=}\PY{n}{job\PYZus{}id}\PY{p}{)}
    \PY{n}{T\PYZus{}SI} \PY{o}{=} \PY{n+nb}{int}\PY{p}{(}\PY{n}{kelvin\PYZus{}from\PYZus{}reduced}\PY{p}{(}\PY{n}{job}\PY{o}{.}\PY{n}{sp}\PY{o}{.}\PY{n}{kT}\PY{p}{[}\PY{l+m+mi}{0}\PY{p}{]}\PY{p}{,}\PY{n}{ref\PYZus{}energy}\PY{p}{)}\PY{p}{)}
    \PY{n+nb}{print}\PY{p}{(}\PY{l+s+sa}{f}\PY{l+s+s2}{\PYZdq{}}\PY{l+s+s2}{e\PYZus{}factor: }\PY{l+s+si}{\PYZob{}}\PY{n}{job}\PY{o}{.}\PY{n}{sp}\PY{o}{.}\PY{n}{e\PYZus{}factor}\PY{l+s+si}{\PYZcb{}}\PY{l+s+s2}{, T: }\PY{l+s+si}{\PYZob{}}\PY{n}{T\PYZus{}SI}\PY{l+s+si}{\PYZcb{}}\PY{l+s+s2}{K}\PY{l+s+s2}{\PYZdq{}}\PY{p}{)}
    \PY{n+nb}{print}\PY{p}{(}\PY{l+s+sa}{f}\PY{l+s+s2}{\PYZdq{}}\PY{l+s+se}{\PYZbs{}t}\PY{l+s+si}{\PYZob{}}\PY{n}{job}\PY{l+s+si}{\PYZcb{}}\PY{l+s+s2}{\PYZdq{}}\PY{p}{)}
\end{Verbatim}
\end{tcolorbox}

    \begin{Verbatim}[commandchars=\\\{\}]
Higest order
e\_factor: 0.4, T: 188K
        6289cb58ab91a927f702b7cf238820aa
e\_factor: 0.4, T: 251K
        16db9573e2e2c6a027abad40cf87733e
e\_factor: 0.2, T: 125K
        c1e543f38d9f49677b3e3649018152b8


Lowest order
e\_factor: 0.2, T: 629K
        c0fd30e6247f95e9c26984ec5f24d99e
e\_factor: 0.4, T: 629K
        e8e42f76f02e02e686332ce8127e1b24
e\_factor: 0.2, T: 503K
        349a22231886a7a46aeb7f20c95204e0
e\_factor: 0.6, T: 629K
        d2da0d6355f3d11fd0132bcf8433df7e
    \end{Verbatim}

    First let's look at one of the high order jobs (e\_factor: 0.4, T:
251K). We'll first visualize the centers of the thiophenes. (It is a
little less busy than the atomistic sturcture and allows us to more
easily see the lamellar spacing.)

The camera position was chosen using the GIXStapose GUI to rotate the
structure as to best see the lamellae.

    \begin{tcolorbox}[breakable, size=fbox, boxrule=1pt, pad at break*=1mm,colback=cellbackground, colframe=cellborder]
\prompt{In}{incolor}{7}{\boxspacing}
\begin{Verbatim}[commandchars=\\\{\}]
\PY{n}{job} \PY{o}{=} \PY{n}{p}\PY{o}{.}\PY{n}{open\PYZus{}job}\PY{p}{(}\PY{n+nb}{id}\PY{o}{=}\PY{l+s+s1}{\PYZsq{}}\PY{l+s+s1}{16db9573e2e2c6a027abad40cf87733e}\PY{l+s+s1}{\PYZsq{}}\PY{p}{)}
\PY{n}{cg\PYZus{}gsd} \PY{o}{=} \PY{n}{job}\PY{o}{.}\PY{n}{fn}\PY{p}{(}\PY{l+s+s2}{\PYZdq{}}\PY{l+s+s2}{cg\PYZhy{}trajectory.gsd}\PY{l+s+s2}{\PYZdq{}}\PY{p}{)}
\PY{n}{scene}\PY{p}{,} \PY{n}{info} \PY{o}{=} \PY{n}{get\PYZus{}scene}\PY{p}{(}\PY{n}{cg\PYZus{}gsd}\PY{p}{,} \PY{n}{color}\PY{o}{=}\PY{p}{\PYZob{}}\PY{l+s+s2}{\PYZdq{}}\PY{l+s+s2}{\PYZus{}B}\PY{l+s+s2}{\PYZdq{}}\PY{p}{:} \PY{l+s+s2}{\PYZdq{}}\PY{l+s+s2}{lightblue}\PY{l+s+s2}{\PYZdq{}}\PY{p}{\PYZcb{}}\PY{p}{,} \PY{n}{scale}\PY{o}{=}\PY{n}{ref\PYZus{}distance}\PY{p}{)}
\PY{n}{cam} \PY{o}{=} \PY{n}{camera}\PY{o}{.}\PY{n}{Orthographic}\PY{p}{(}
    \PY{n}{position} \PY{o}{=} \PY{p}{[}\PY{l+m+mf}{8.133}\PY{p}{,} \PY{l+m+mf}{5.203}\PY{p}{,} \PY{l+m+mf}{43.865}\PY{p}{]}\PY{p}{,}
    \PY{n}{look\PYZus{}at} \PY{o}{=}  \PY{p}{[}\PY{l+m+mf}{0.000}\PY{p}{,} \PY{l+m+mf}{0.000}\PY{p}{,} \PY{l+m+mf}{0.000}\PY{p}{]}\PY{p}{,}
    \PY{n}{up} \PY{o}{=}       \PY{p}{[}\PY{l+m+mf}{0.900}\PY{p}{,} \PY{l+m+mf}{0.424}\PY{p}{,} \PY{o}{\PYZhy{}}\PY{l+m+mf}{0.101}\PY{p}{]}\PY{p}{,}
    \PY{n}{height} \PY{o}{=}   \PY{l+m+mf}{31.896}
\PY{p}{)}
\PY{n}{scene}\PY{o}{.}\PY{n}{camera} \PY{o}{=} \PY{n}{cam}

\PY{n}{scene}\PY{o}{.}\PY{n}{geometry}\PY{p}{[}\PY{l+m+mi}{0}\PY{p}{]}\PY{o}{.}\PY{n}{radius}\PY{p}{[}\PY{p}{:}\PY{p}{]} \PY{o}{*}\PY{o}{=} \PY{l+m+mi}{3}

\PY{n}{output} \PY{o}{=} \PY{n}{pathtrace}\PY{p}{(}\PY{n}{scene}\PY{p}{,} \PY{n}{light\PYZus{}samples}\PY{o}{=}\PY{l+m+mi}{40}\PY{p}{,} \PY{n}{w}\PY{o}{=}\PY{l+m+mi}{600}\PY{p}{,} \PY{n}{h}\PY{o}{=}\PY{l+m+mi}{600}\PY{p}{)}

\PY{n}{image} \PY{o}{=} \PY{n}{Image}\PY{o}{.}\PY{n}{fromarray}\PY{p}{(}\PY{n}{output}\PY{p}{[}\PY{p}{:}\PY{p}{]}\PY{p}{,} \PY{n}{mode}\PY{o}{=}\PY{l+s+s2}{\PYZdq{}}\PY{l+s+s2}{RGBA}\PY{l+s+s2}{\PYZdq{}}\PY{p}{)}
\PY{n}{image}\PY{o}{.}\PY{n}{save}\PY{p}{(}\PY{l+s+s2}{\PYZdq{}}\PY{l+s+s2}{cg\PYZhy{}trajectory\PYZus{}scene.png}\PY{l+s+s2}{\PYZdq{}}\PY{p}{)}

\PY{n}{output}
\end{Verbatim}
\end{tcolorbox}
 
            
\prompt{Out}{outcolor}{7}{}
    
    \begin{center}
    \adjustimage{max size={0.9\linewidth}{0.9\paperheight}}{figures/opa_notebook/Order_parameter_analysis_13_0.png}
    \end{center}
    

    We can see that this high order structure has clear repeated planes of
thiophenes, even within these planes there appears to be some ordering
most likely due to pi orbital overlap in the thiophenes. Let's look at
the diffraction pattern of this view:

    \begin{tcolorbox}[breakable, size=fbox, boxrule=1pt, pad at break*=1mm,colback=cellbackground, colframe=cellborder]
\prompt{In}{incolor}{8}{\boxspacing}
\begin{Verbatim}[commandchars=\\\{\}]
\PY{n}{d} \PY{o}{=} \PY{n}{Diffractometer}\PY{p}{(}\PY{n}{length\PYZus{}scale}\PY{o}{=}\PY{n+nb}{float}\PY{p}{(}\PY{n}{ref\PYZus{}distance}\PY{p}{)}\PY{p}{)}
\PY{n}{d}\PY{o}{.}\PY{n}{load}\PY{p}{(}\PY{n}{info}\PY{p}{[}\PY{l+s+s2}{\PYZdq{}}\PY{l+s+s2}{positions}\PY{l+s+s2}{\PYZdq{}}\PY{p}{]}\PY{p}{,} \PY{n}{info}\PY{p}{[}\PY{l+s+s2}{\PYZdq{}}\PY{l+s+s2}{box}\PY{l+s+s2}{\PYZdq{}}\PY{p}{]}\PY{p}{[}\PY{p}{:}\PY{l+m+mi}{3}\PY{p}{]}\PY{p}{)}
\PY{n}{d}\PY{o}{.}\PY{n}{diffract\PYZus{}from\PYZus{}camera}\PY{p}{(}\PY{n}{cam}\PY{p}{)}
\PY{n}{fig}\PY{p}{,} \PY{n}{ax} \PY{o}{=} \PY{n}{d}\PY{o}{.}\PY{n}{plot}\PY{p}{(}\PY{n}{cmap}\PY{o}{=}\PY{l+s+s2}{\PYZdq{}}\PY{l+s+s2}{jet}\PY{l+s+s2}{\PYZdq{}}\PY{p}{,} \PY{n}{crop}\PY{o}{=}\PY{l+m+mf}{2.6}\PY{p}{)}
\PY{n}{y\PYZus{}min}\PY{p}{,} \PY{n}{y\PYZus{}max} \PY{o}{=} \PY{n}{ax}\PY{o}{.}\PY{n}{get\PYZus{}ylim}\PY{p}{(}\PY{p}{)}
\PY{n}{ax}\PY{o}{.}\PY{n}{set\PYZus{}ylim}\PY{p}{(}\PY{p}{(}\PY{o}{\PYZhy{}}\PY{l+m+mf}{0.01}\PY{p}{,} \PY{n}{y\PYZus{}max}\PY{p}{)}\PY{p}{)}
\PY{p}{[}\PY{n}{l}\PY{o}{.}\PY{n}{set\PYZus{}visible}\PY{p}{(}\PY{k+kc}{False}\PY{p}{)} \PY{k}{for} \PY{n}{l} \PY{o+ow}{in} \PY{n}{ax}\PY{o}{.}\PY{n}{xaxis}\PY{o}{.}\PY{n}{get\PYZus{}ticklabels}\PY{p}{(}\PY{p}{)}\PY{p}{[}\PY{p}{:}\PY{p}{:}\PY{l+m+mi}{2}\PY{p}{]}\PY{p}{]}
\PY{p}{[}\PY{n}{l}\PY{o}{.}\PY{n}{set\PYZus{}visible}\PY{p}{(}\PY{k+kc}{False}\PY{p}{)} \PY{k}{for} \PY{n}{l} \PY{o+ow}{in} \PY{n}{ax}\PY{o}{.}\PY{n}{yaxis}\PY{o}{.}\PY{n}{get\PYZus{}ticklabels}\PY{p}{(}\PY{p}{)}\PY{p}{[}\PY{p}{:}\PY{p}{:}\PY{l+m+mi}{2}\PY{p}{]}\PY{p}{]}

\PY{n}{plt}\PY{o}{.}\PY{n}{show}\PY{p}{(}\PY{p}{)}
\end{Verbatim}
\end{tcolorbox}

    \begin{center}
    \adjustimage{max size={0.9\linewidth}{0.9\paperheight}}{figures/opa_notebook/Order_parameter_analysis_15_0.png}
    \end{center}
    
    In this view, we can see bright peaks in the \(q_{z}\)-direction
corresponding to the lamellar spacing. The pi-stacking can also faintly
be seen in the \(q_{xy}\)-direction. We can label these peaks with their
distance from the origin using the interactive PeakLabeller class
provided in GIXStapose. (To see a demo of how to use this class check
out the
\href{https://github.com/cmelab/GIXStapose/blob/master/examples/Figure_Example.ipynb}{GIXStapose
example}.)

    \begin{center}
    \adjustimage{max size={0.9\linewidth}{0.9\paperheight}}{figures/opa_notebook/resize_labelled_peaks.png}
    \end{center}

These peaks correspond to periodic distances of 17.24 \AA~ and 3.61 \AA~
which are in the right ballpark for P3HT:
\href{https://www.sciencedirect.com/science/article/pii/S1566119913000840?via\%3Dihub}{Duong
2013} reports a lamellar spacing of 16.5 \AA~ and a pi-stacking spacing of
3.83 \AA~ for neat P3HT.

In the next cell, we'll visualize the overlay of the coarse grain
structure with the atomistic one:

    \begin{tcolorbox}[breakable, size=fbox, boxrule=1pt, pad at break*=1mm,colback=cellbackground, colframe=cellborder]
\prompt{In}{incolor}{9}{\boxspacing}
\begin{Verbatim}[commandchars=\\\{\}]
\PY{n}{ua\PYZus{}gsd} \PY{o}{=} \PY{n}{job}\PY{o}{.}\PY{n}{fn}\PY{p}{(}\PY{l+s+s2}{\PYZdq{}}\PY{l+s+s2}{trajectory.gsd}\PY{l+s+s2}{\PYZdq{}}\PY{p}{)}
\PY{n}{ua\PYZus{}scene}\PY{p}{,} \PY{n}{\PYZus{}} \PY{o}{=} \PY{n}{get\PYZus{}scene}\PY{p}{(}
    \PY{n}{ua\PYZus{}gsd}\PY{p}{,} 
    \PY{n}{color}\PY{o}{=}\PY{p}{\PYZob{}}\PY{l+s+s1}{\PYZsq{}}\PY{l+s+s1}{c3}\PY{l+s+s1}{\PYZsq{}}\PY{p}{:} \PY{l+s+s2}{\PYZdq{}}\PY{l+s+s2}{grey}\PY{l+s+s2}{\PYZdq{}}\PY{p}{,} \PY{l+s+s1}{\PYZsq{}}\PY{l+s+s1}{cc}\PY{l+s+s1}{\PYZsq{}}\PY{p}{:} \PY{l+s+s2}{\PYZdq{}}\PY{l+s+s2}{grey}\PY{l+s+s2}{\PYZdq{}}\PY{p}{,} \PY{l+s+s1}{\PYZsq{}}\PY{l+s+s1}{cd}\PY{l+s+s1}{\PYZsq{}}\PY{p}{:} \PY{l+s+s2}{\PYZdq{}}\PY{l+s+s2}{grey}\PY{l+s+s2}{\PYZdq{}}\PY{p}{,} \PY{l+s+s1}{\PYZsq{}}\PY{l+s+s1}{ss}\PY{l+s+s1}{\PYZsq{}}\PY{p}{:} \PY{l+s+s2}{\PYZdq{}}\PY{l+s+s2}{yellow}\PY{l+s+s2}{\PYZdq{}}\PY{p}{\PYZcb{}}\PY{p}{,} 
    \PY{n}{scale}\PY{o}{=}\PY{n}{ref\PYZus{}distance}\PY{p}{,}
    \PY{n}{scene}\PY{o}{=}\PY{n}{scene}
\PY{p}{)}

\PY{n}{ua\PYZus{}scene}\PY{o}{.}\PY{n}{camera} \PY{o}{=} \PY{n}{cam}

\PY{n}{ua\PYZus{}scene}\PY{o}{.}\PY{n}{geometry}\PY{p}{[}\PY{l+m+mi}{0}\PY{p}{]}\PY{o}{.}\PY{n}{material}\PY{o}{.}\PY{n}{spec\PYZus{}trans} \PY{o}{=} \PY{l+m+mf}{0.5}

\PY{n}{ua\PYZus{}scene}\PY{o}{.}\PY{n}{lights} \PY{o}{=} \PY{n}{light}\PY{o}{.}\PY{n}{cloudy}\PY{p}{(}\PY{p}{)}

\PY{n}{output} \PY{o}{=} \PY{n}{pathtrace}\PY{p}{(}\PY{n}{ua\PYZus{}scene}\PY{p}{,} \PY{n}{light\PYZus{}samples}\PY{o}{=}\PY{l+m+mi}{40}\PY{p}{,} \PY{n}{w}\PY{o}{=}\PY{l+m+mi}{600}\PY{p}{,} \PY{n}{h}\PY{o}{=}\PY{l+m+mi}{600}\PY{p}{)}

\PY{n}{image} \PY{o}{=} \PY{n}{Image}\PY{o}{.}\PY{n}{fromarray}\PY{p}{(}\PY{n}{output}\PY{p}{[}\PY{p}{:}\PY{p}{]}\PY{p}{,} \PY{n}{mode}\PY{o}{=}\PY{l+s+s2}{\PYZdq{}}\PY{l+s+s2}{RGBA}\PY{l+s+s2}{\PYZdq{}}\PY{p}{)}
\PY{n}{image}\PY{o}{.}\PY{n}{save}\PY{p}{(}\PY{l+s+s2}{\PYZdq{}}\PY{l+s+s2}{cg\PYZhy{}overlay\PYZus{}scene.png}\PY{l+s+s2}{\PYZdq{}}\PY{p}{)}

\PY{n}{output}
\end{Verbatim}
\end{tcolorbox}
 
            
\prompt{Out}{outcolor}{9}{}
    
    \begin{center}
    \adjustimage{max size={0.9\linewidth}{0.9\paperheight}}{figures/opa_notebook/Order_parameter_analysis_17_0.png}
    \end{center}
    

    When we view this overlay (thiophene center beads show in translucent
blue, atomistic carbons in grey and sulfur in yellow), it becomes more
clear that this lamellar spacing is due to the regions with the alkyl
tails interacting.

Finally let's look at one of the low order jobs (e\_factor: 0.2, T:
629K):

    \begin{tcolorbox}[breakable, size=fbox, boxrule=1pt, pad at break*=1mm,colback=cellbackground, colframe=cellborder]
\prompt{In}{incolor}{10}{\boxspacing}
\begin{Verbatim}[commandchars=\\\{\}]
\PY{n}{job} \PY{o}{=} \PY{n}{p}\PY{o}{.}\PY{n}{open\PYZus{}job}\PY{p}{(}\PY{n+nb}{id}\PY{o}{=}\PY{l+s+s1}{\PYZsq{}}\PY{l+s+s1}{c0fd30e6247f95e9c26984ec5f24d99e}\PY{l+s+s1}{\PYZsq{}}\PY{p}{)}
\PY{n}{cg\PYZus{}gsd} \PY{o}{=} \PY{n}{job}\PY{o}{.}\PY{n}{fn}\PY{p}{(}\PY{l+s+s2}{\PYZdq{}}\PY{l+s+s2}{cg\PYZhy{}trajectory.gsd}\PY{l+s+s2}{\PYZdq{}}\PY{p}{)}
\PY{n}{scene}\PY{p}{,} \PY{n}{info} \PY{o}{=} \PY{n}{get\PYZus{}scene}\PY{p}{(}\PY{n}{cg\PYZus{}gsd}\PY{p}{,} \PY{n}{color}\PY{o}{=}\PY{p}{\PYZob{}}\PY{l+s+s2}{\PYZdq{}}\PY{l+s+s2}{\PYZus{}B}\PY{l+s+s2}{\PYZdq{}}\PY{p}{:} \PY{l+s+s2}{\PYZdq{}}\PY{l+s+s2}{lightblue}\PY{l+s+s2}{\PYZdq{}}\PY{p}{\PYZcb{}}\PY{p}{,} \PY{n}{scale}\PY{o}{=}\PY{n}{ref\PYZus{}distance}\PY{p}{)}

\PY{n}{scene}\PY{o}{.}\PY{n}{geometry}\PY{p}{[}\PY{l+m+mi}{0}\PY{p}{]}\PY{o}{.}\PY{n}{radius}\PY{p}{[}\PY{p}{:}\PY{p}{]} \PY{o}{*}\PY{o}{=} \PY{l+m+mi}{3}

\PY{n}{output} \PY{o}{=} \PY{n}{pathtrace}\PY{p}{(}\PY{n}{scene}\PY{p}{,} \PY{n}{light\PYZus{}samples}\PY{o}{=}\PY{l+m+mi}{40}\PY{p}{,} \PY{n}{w}\PY{o}{=}\PY{l+m+mi}{600}\PY{p}{,} \PY{n}{h}\PY{o}{=}\PY{l+m+mi}{600}\PY{p}{)}

\PY{n}{image} \PY{o}{=} \PY{n}{Image}\PY{o}{.}\PY{n}{fromarray}\PY{p}{(}\PY{n}{output}\PY{p}{[}\PY{p}{:}\PY{p}{]}\PY{p}{,} \PY{n}{mode}\PY{o}{=}\PY{l+s+s2}{\PYZdq{}}\PY{l+s+s2}{RGBA}\PY{l+s+s2}{\PYZdq{}}\PY{p}{)}
\PY{n}{image}\PY{o}{.}\PY{n}{save}\PY{p}{(}\PY{l+s+s2}{\PYZdq{}}\PY{l+s+s2}{cg\PYZhy{}trajectory\PYZhy{}amorphous\PYZus{}scene.png}\PY{l+s+s2}{\PYZdq{}}\PY{p}{)}

\PY{n}{output}
\end{Verbatim}
\end{tcolorbox}
 
            
\prompt{Out}{outcolor}{10}{}
    
    \begin{center}
    \adjustimage{max size={0.9\linewidth}{0.9\paperheight}}{figures/opa_notebook/Order_parameter_analysis_19_0.png}
    \end{center}
    

    This view shows that the packing looks basically random.

And we can confirm this by examining the diffraction pattern:

    \begin{tcolorbox}[breakable, size=fbox, boxrule=1pt, pad at break*=1mm,colback=cellbackground, colframe=cellborder]
\prompt{In}{incolor}{11}{\boxspacing}
\begin{Verbatim}[commandchars=\\\{\}]
\PY{n}{d} \PY{o}{=} \PY{n}{Diffractometer}\PY{p}{(}\PY{n}{length\PYZus{}scale}\PY{o}{=}\PY{n+nb}{float}\PY{p}{(}\PY{n}{ref\PYZus{}distance}\PY{p}{)}\PY{p}{)}
\PY{n}{d}\PY{o}{.}\PY{n}{load}\PY{p}{(}\PY{n}{info}\PY{p}{[}\PY{l+s+s2}{\PYZdq{}}\PY{l+s+s2}{positions}\PY{l+s+s2}{\PYZdq{}}\PY{p}{]}\PY{p}{,} \PY{n}{info}\PY{p}{[}\PY{l+s+s2}{\PYZdq{}}\PY{l+s+s2}{box}\PY{l+s+s2}{\PYZdq{}}\PY{p}{]}\PY{p}{[}\PY{p}{:}\PY{l+m+mi}{3}\PY{p}{]}\PY{p}{)}
\PY{n}{d}\PY{o}{.}\PY{n}{diffract\PYZus{}from\PYZus{}camera}\PY{p}{(}\PY{n}{cam}\PY{p}{)}
\PY{n}{fig}\PY{p}{,} \PY{n}{ax} \PY{o}{=} \PY{n}{d}\PY{o}{.}\PY{n}{plot}\PY{p}{(}\PY{n}{cmap}\PY{o}{=}\PY{l+s+s2}{\PYZdq{}}\PY{l+s+s2}{jet}\PY{l+s+s2}{\PYZdq{}}\PY{p}{,} \PY{n}{crop}\PY{o}{=}\PY{l+m+mf}{2.6}\PY{p}{)}
\PY{n}{y\PYZus{}min}\PY{p}{,} \PY{n}{y\PYZus{}max} \PY{o}{=} \PY{n}{ax}\PY{o}{.}\PY{n}{get\PYZus{}ylim}\PY{p}{(}\PY{p}{)}
\PY{n}{ax}\PY{o}{.}\PY{n}{set\PYZus{}ylim}\PY{p}{(}\PY{p}{(}\PY{o}{\PYZhy{}}\PY{l+m+mf}{0.01}\PY{p}{,} \PY{n}{y\PYZus{}max}\PY{p}{)}\PY{p}{)}
\PY{p}{[}\PY{n}{l}\PY{o}{.}\PY{n}{set\PYZus{}visible}\PY{p}{(}\PY{k+kc}{False}\PY{p}{)} \PY{k}{for} \PY{n}{l} \PY{o+ow}{in} \PY{n}{ax}\PY{o}{.}\PY{n}{xaxis}\PY{o}{.}\PY{n}{get\PYZus{}ticklabels}\PY{p}{(}\PY{p}{)}\PY{p}{[}\PY{p}{:}\PY{p}{:}\PY{l+m+mi}{2}\PY{p}{]}\PY{p}{]}
\PY{p}{[}\PY{n}{l}\PY{o}{.}\PY{n}{set\PYZus{}visible}\PY{p}{(}\PY{k+kc}{False}\PY{p}{)} \PY{k}{for} \PY{n}{l} \PY{o+ow}{in} \PY{n}{ax}\PY{o}{.}\PY{n}{yaxis}\PY{o}{.}\PY{n}{get\PYZus{}ticklabels}\PY{p}{(}\PY{p}{)}\PY{p}{[}\PY{p}{:}\PY{p}{:}\PY{l+m+mi}{2}\PY{p}{]}\PY{p}{]}

\PY{n}{plt}\PY{o}{.}\PY{n}{show}\PY{p}{(}\PY{p}{)}
\end{Verbatim}
\end{tcolorbox}

    \begin{center}
    \adjustimage{max size={0.9\linewidth}{0.9\paperheight}}{figures/opa_notebook/Order_parameter_analysis_21_0.png}
    \end{center}
    
    The general lack of peaks suggests that there are few periodic features
to be found.

In conclusion, this analysis has shown that we can reproduce our prior
work with updated tools, the order parameter metric is robust across
different forcefields and relates to the prominence of peaks in the
diffraction pattern.

    \begin{tcolorbox}[breakable, size=fbox, boxrule=1pt, pad at break*=1mm,colback=cellbackground, colframe=cellborder]
\prompt{In}{incolor}{ }{\boxspacing}
\begin{Verbatim}[commandchars=\\\{\}]

\end{Verbatim}
\end{tcolorbox}

